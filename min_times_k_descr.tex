\documentclass[a4paper]{article}
\usepackage{fullpage}

\usepackage{algpseudocode}
\usepackage{algorithmicx}

\usepackage{comment}
\usepackage{multirow}


\title{min-times-k algorithm}
\author{Guillermo Montero Melis \and Jaime Melis Bayo}

\begin{document}
\maketitle

\section{Introduction}

This is a high-level description of the \verb:min_times_k: algorithm.
The description is intended to be understood by humans, so as to then be translated to a programming language (e.g. Python).
The question the algorithm attempts to answer is:
How many subgroups of size $k$ do we minimally need so that all possible pairs of elements in $n$ appear at least once in one of the subgroups? (NB: subgroups allow no repetition.)

\subsection{Pedagogical analogy}
\label{ssec:pedag}

It is analogous to the following situation:
There is a group of $n$ people waiting outside of a room.
The goal is that all people greet each other at least once.
However, for some reason these people cannot greet each other outside of the room (the corridor is too narrow, say). So they have to come into the room in order to greet.
But the room is not big enough for all n people to fit in! So subsets of $n$ of size $k$ people ($k \leq n$) have to come in each time.
Once a group comes into the room, all people greet each other, even if they have already greeted.

The question is then: How many groups will we have to form in order for everybody to greet each other at least once?
The algorithm is designed to minimize the number of subgroups of size $k$ we will choose.
Hence its name \verb:min_times_k: algorithm.
It even does a couple of things more; read on.

\subsection{What the algorithm does}

This is what the \verb:min_times_k: algorithm does:

\begin{itemize}
\item Find out the minimum number of subgroups of size $k$ needed for all $n$ elements to have appeared at least once together in a subgroup.
\item Minimize the number of repetitions of the same pair, i.e. we rather have one repetition of four different pairs than four repetitions of the same pair.
\item Produce a list of the items in each subgroup, i.e. it has to be designed as a program that will tell us which element to pick each time.
\end{itemize}

\subsection{Input and output of the eventual program}

Even if the main goal of this document is to present the algorithm, it is also a guide for its implementation as a program, which will crucially take some input and produce some output.
The program will take as its input two positive integers: $ n \geq 2$ and $k \leq n$.
It will output the following:

\begin{itemize}
\item[min-times-k:]
The minimum amount of subgroups size $k$ needed for all pairs in $n$ to appear together at least once.
\item[rep\_pairs:]
A matrix showing the frequency with which each pair has occurred.
\item[R:] 
An array with the index of the elements chosen in each of the subgroups of size $k$.
\end{itemize}

\section{Notation}

When referring to an object we will call it by its one-letter-shorthand, e.g. $O$, $P$ or $R$ (see section~\ref{sec:objects}).
As these objects can be represented as matrices we will index the element in row $i$ and column $j$ of the object $Object$ as $Object[i,j]$.
Note that we start counting rows and columns with $0$,
which means that the first element of $Object$ situated in the first row (row 0) and the first column (column 0) is $Object[0,0]$.
Row 0 of the same object is $Object[0,]$ and column 1 is $Object[,1]$.

\section{Objects}
\label{sec:objects}

The program contains three matrix-type objects, which we call $O$ (for occurrences),
$P$ (for pairs), 
and $R$ (for results).
They are described in turn.

\subsection{O: occurrences}
\label{ssec:occurr}

This object is a two-dimensional matrix with two rows and $n$ columns.
$O$ tracks two counts of occurrences associated to each of the $n$ elements.
Table~\ref{tab:O_0} shows the initial state of $O$.
Of course, all the occurrences are equal to $0$ at this point.
Note that $O$ has two rows:
In row 0 of $O$ ($O[0,]$), all the occurrences of element $i$ in $R$ (see section~\ref{ssect:R}) are counted.
In row 1 ($O[1,]$) we count in how many of the $n-1$ pairs in which an element can appear, it has already appeared.
(NB: We do not count the pair of an element with itself.)

\begin{table}
\centering
\begin{tabular} {| l ||c |c |c |c |c |c |c |c |c |c |}  
\multicolumn{ 11 }{c}{O (occurrences) } \\  
\multicolumn{1}{l}{index} & \multicolumn{1}{c}{0} & \multicolumn{1}{c}{1} & \multicolumn{1}{c}{2} & \multicolumn{1}{c}{3} & \multicolumn{1}{c}{4} & \multicolumn{1}{c}{5} & \multicolumn{1}{c}{6} & \multicolumn{1}{c}{7} & \multicolumn{1}{c}{\ldots} & \multicolumn{1}{c}{$n-1$} \\ \hline 
occurr & 0 & 0 & 0 & 0 & 0 & 0 & 0 & 0 & \ldots & 0 \\ \hline 
occurrInPairs & 0 & 0 & 0 & 0 & 0 & 0 & 0 & 0 & \ldots & 0 \\ \hline 
\end{tabular}  
\caption{
Row 1 (\emph{occurr}) shows how often element $i$ has occurred in a subgroup;
row 2 (\emph{occurrInPairs}) shows how many pairs have been formed with $i$.
(Note that the index is included in this table only for the convenience of the reader; however, it is not part of the object and therefore we do not count it as a row.)
}
\label{tab:O_0}
\end{table}

This will become clearer with an example.
For instance, let's take $n=4$, so our elements are $\{0,1,2,3\}$.
Say we have formed the subgroups $A=\{0,1,2\}$ and $B=\{1,2,3\}$ so far.
The state of object $O$ is now shown in table~\ref{tab:O_1}.
Row 0 (``occurr") shows the occurrences of each element; this is straightforward.
Row 1 (``occurInPairs'') shows the number of \emph{different} pairs in which each element has occurred.
$0$ has occurred with $1$ and $2$ in $A$, and since $0$ has not occurred in B, it has occurred with two pairs, and that's the value appearing in the corresponding row of table~\ref{tab:O_1}.
The element $1$ has occurred with $0$ and $2$ in $A$ (i.e. with two different pairs); and additionally it has appeared with $2$ and $3$ in $B$.
It has thus appeared in three \emph{different} pairs.
In other words, we count the pair $\{1,2\}$ only once.
This is shown in table~\ref{tab:O_1}, and the same reasoning applies for the other elements.

\begin{table}
\centering
\begin{tabular} {| l ||c |c |c |c |}  
\multicolumn{ 5 }{c}{O (occurrences) } \\  
\multicolumn{1}{l}{index} & \multicolumn{1}{c}{0} & \multicolumn{1}{c}{1} & \multicolumn{1}{c}{2} & \multicolumn{1}{c}{3} \\ \hline 
occurr & 1 & 2 & 2 & 1 \\ \hline 
occurrInPairs & 2 & 3 & 3 & 2  \\ \hline 
\end{tabular}  
\caption{
Occurrences in the hypothetical example mentioned in section~\ref{ssec:occurr}.
}
\label{tab:O_1}
\end{table}


\subsection{P: pairs}

Matrix $P$ shows all possible pairings and their respective occurrences (see table~\ref{tab:P_0}). 
$P[0,]$ (\emph{pairs}) shows all possible pairings between the n elements; each element is represented by its index in $O$. 
$P[1,]$ (\emph{occurr}) keeps track of how often this pair has occurred in a subgroup.

\begin{table}
\centering
\begin{tabular} {| l ||c |c |c |c |c |c |c |c |c |c |}  
\multicolumn{ 11 }{c}{ P (pairs) } \\ \hline 
pair & 0,1 & 0,2 & 0,3 & \ldots & 0,$n-1$ & 1,2 & \ldots & 1,$n-1$ & \ldots & $n-2$,$n-1$ \\ \hline 
occurr & 0 & 0 & 0 & \ldots & 0 & 0 & \ldots & 0 & \ldots & 0 \\ \hline 
\end{tabular}  
\caption{Possible pairs and their occurrences.
}
\label{tab:P_0}
\end{table}


\subsection{R: results}
\label{ssect:R}

The array $R$ (results) keeps track of the elements picked out by the algorithm: each column represents a subgroup.
Keeping the analogy above (see section~\ref{ssec:pedag}), each column in $R$ shows what people enter the room each time.
Since we want to form subgroups of size $k$ elements, each column in $R$ will comprise $k$ rows.
The number of columns is precisely what we are looking for -- it is equal to \verb:min-size-k:.
Table~\ref{tab:R_0} shows the initial state of this array for $n$ and $k$ elements.
\emph{NA} denotes that the values are not available, since the algorithm has not yet produced any values.
Note also that the number of columns in table~\ref{tab:R_0} is arbitrary -- once this algorithm is implemented the program will just need to start out with one, and add one whenever necessary.

\begin{table}
\centering
\begin{tabular} {| l ||c |c |c |}  
\multicolumn{ 4 }{c}{ R (results) } \\ \hline 
   \ index  & subgroup1 & subgroup2 & \ldots \\ \hline \hline 
0 & NA & NA & \ldots \\ \hline 
1 & NA & NA & \ldots\\ \hline 
2 & NA & NA & \ldots\\ \hline 
\ldots & \ldots & \ldots & \ldots \\ \hline 
$k-1$ & NA & NA & \ldots\\ \hline 
\end{tabular} 
\caption{Elements picked by the algorithm in each subgroup}
\label{tab:R_0}
\end{table}


\section{High-level description of min-times-k algorithm}

\begin{enumerate}

\item\label{alg:findnext}
	\verb!FindNext_i!: Find first element which has the lowest value in $O[1,]$ and assign its index value to $i$

\item
	\verb!print_i!: Print $i$ in next empty cell of $R$ (read $R$ columnwise)

\item\label{alg:check_R}
	\verb!check_R!:

	\begin{algorithmic}
	\If{current column of $R$ is not full}
		\State GOTO~\ref{alg:findpair}
	\EndIf
	\end{algorithmic}

\item\label{alg:check_stop} 
	\verb!check_stop!:

	\begin{algorithmic}
		\If{for all elements $e$, $A[1,e] = n-1$}
			\State STOP
		\EndIf
	\end{algorithmic}

\item\label{alg:findpair}
	\verb!FindPair!:
	
	\begin{algorithmic}
	\If{there is an element which has not been paired with $i$ AND which has the lowest value in $O[0,]$}
		\State Assign that value to $j$
		\State GOTO~\ref{alg:print_j}
	\ElsIf{there is an element which has not been paired with $i$ AND which has the lowest value in $O[0,]$ plus $1$}
		\State Assign that value to $j$
		\State GOTO~\ref{alg:print_j}
	\ElsIf{there is an element which has not been paired with $i$ AND which has the lowest value in $O[0,]$ plus $2$ etc. (loop until $max(O[0,])$ is reached)}
		\State Assign that value to $j$
		\State GOTO~\ref{alg:print_j}
	\ElsIf{$i$ has been paired with all other elements}
		\State Assign the value of the preceding element in $R$ to $i$ (one cell up)
		\State GOTO~\ref{alg:findpair}
	\Else
		\State GOTO~\ref{alg:findnext}
	\EndIf
	\end{algorithmic}

	\item \verb!assign_j!: $j \leftarrow$ value found in~\ref{alg:findpair} 

\item\label{alg:print_j} 
	\verb!print_j!: Print $j$ in $R$

\item\label{alg:check_R_bis}
	\verb!check_R!:

	\begin{algorithmic}
	\If{current column of $R$ is not full}
		\State GOTO~\ref{alg:i=j}
	\EndIf
	\end{algorithmic}

\item\label{alg:check_stop_bis} 
	\verb!check_stop!:

	\begin{algorithmic}
		\If{for all $i$, $A[1,i] = n-1$}
			\State STOP
		\EndIf
	\end{algorithmic}

\item\label{alg:i=j} 
	\verb!i=j!: $i \leftarrow j$

\item GOTO~\ref{alg:findpair}

\end{enumerate}

% The whole next section is not included (marked as commented. We'll see if it's necessary to add an example since it is so much work creating all the tables.
\begin{comment}
\section{Example with $n=7, k=4$}

The initial state of $O$, $P$ and $R$ is shown in tables~\ref{tab:O-ex00} through~\ref
We begin the 

\begin{table}
\centering
\begin{tabular} {| l ||c |c |c |c |c |c |c |}  
\multicolumn{ 8 }{c}{$O_0$ } \\  
\multicolumn{1}{l}{index} & \multicolumn{1}{c}{0} & \multicolumn{1}{c}{1} & \multicolumn{1}{c}{2} & \multicolumn{1}{c}{3} & \multicolumn{1}{c}{4} & \multicolumn{1}{c}{5} & \multicolumn{1}{c}{6} \\ \hline 
occurr & 0 & 0 & 0 & 0 & 0 & 0 & 0 \\ \hline 
occurrInPairs & 0 & 0 & 0 & 0 & 0 & 0 & 0 \\ \hline 
\end{tabular}  

\begin{tabular} {| l ||c |c |c |c |c |c |c |c |c |c |c |c |c |c |c |c |c |c |c |c |c |c |c |c |}  
\multicolumn{ 25 }{c}{ $P_0$ } \\ \hline 
pair & 0,1 & 0,2 & 0,3 & 0,4 & 0,5 & 0,6 & 1,2 & 1,3 & 1,4 & 1,5 & 1,6 & 2,3 & 2,4 & 2,5 & 2,6 & 3,4 & 3,5 & 3,6 & 4,5 & 4,6 & 5,6 & 7,8 & 7,9 & 8,9 \\ \hline 
occurr & 0 & 0 & 0 & 0 & 0 & 0 & 0 & 0 & 0 & 0 & 0 & 0 & 0 & 0 & 0 & 0 & 0 & 0 & 0 & 0 & 0 & 0 & 0 & 0 \\ \hline 
\end{tabular} 

\caption{
Initial state of $O$, $P$ and $R$ for $n=7, k=4$
}
\label{tab:O-ex00}
\end{table}

\end{comment}
% End of the comment!
 
\section{Implementation of the min-times-k algorithm in Python}

\verb!Almost done...!

% DOn't know whu TeX doesn't like this source code for the algorithm (it works in my office computer version). It's marked as a comment for now.

\begin{comment}
\begin{algorithm}

\begin{algorithmic}[1]% Taken from the algorithmicx package documentation
  \Procedure{Euclid}{$a,b$}
  \State $r\gets a\bmod b$
  \While{$r\not=0$}
    \State $a\gets b$
    \State $b\gets r$
    \State $r\gets a\bmod b$
  \EndWhile\label{euclidendwhile}
  \State \textbf{return} $b$
  \EndProcedure
\end{algorithmic}
\caption{Just some example I found}

\end{algorithm}
\end{comment}


\end{document}